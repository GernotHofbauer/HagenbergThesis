\chapter{Einleitung}
    Um den Abschluss des Fachhochschul-Bachelorstudiengangs Software Engineering an
    der FH Hagenberg zu erlangen, muss im Rahmen des Studiums ein zwölfwöchiges Be-
    rufspraktikum absolviert werden. Dieses Praktikum wird in der Regel im sechsten Se-
    mester absolviert und erfordert die Zusammenarbeit mit einem von der FH geprüften
    Unternehmen oder einer Einrichtung. Anschließend ist dazu ein Bericht abzugeben, der
    die Tätigkeiten und Ergebnisse des Praktikums beschreibt. Der
    Bericht ist vollkommen in deutscher Sprache verfasst.

\section{Unternehmen und Arbeitsumfeld}

    Das Praktikum wird bei \emph{ITPRO - Consulting \& Software GmbH} \cite{ITPRO} absolviert. Dabei handelt es sich um ein Unternehmen, das sich auf die Entwicklung von Softwarelösungen 
    spezialisiert hat. Gegründet wurde es im Jahre 1999 und hat seinen Hauptsitz in Linz. Kleinere Büros befinden sich in Hagenberg und Ottensheim. Derzeit sind insgesamt um die 
    50 Mitarbeiter*innen in den verschiedenen Büros beschäftigt. Diese können bis auf wenige Ausnahmen frei zwischen den Büros wechseln und dort arbeiten. Unter normalen Umständen sind 
    um die 8 Mitarbeiter*innen  aus 2 verschiedenen Teams im hagenberger Standort anzutreffen.
    Der Großteil des Praktikums wurde aus diesem Grund in Hagenberg absolviert.
    ITPRO bietet Außendienst-Lösungen, Dienstleistungs-ERP, Lösungen für \gls{oepnv}-Unternehmen und individuelle Software-Lösungen an für die auch individuelle Hardware entwickelt
    und ausgeliefert wird.

    Das Praktikum sieht eine Stelle als Entwickler im internen Team \emph{"Team Nachtschicht"} vor. Der Name wurde von den Entwicklern gewählt und kann keinerlei tiefere Bedeutung zugeordnet werden.
    Dieses besteht derzeit aus 11 Personen und ist für die Entwicklung 
    von \gls{oepnv}-Lösungen zuständig. Als Vorgehensmodell des Projektmanagements wird Scrum verwendet. Die Technologien werden auf Absprache mit dem Kunden ausgewählt. Gibt es von deren Seite keine
    Preferenz, so wird in in der Regel auf C\# und .NET zurückgegriffen.

    Obwohl die Teilnahme an Scrum-Meetings verpflichtend war, wurden mit nur wenigen Ausnahmen ausschließlich mit 2 Mitarbeitern zusammengearbeitet. Dabei handelte es sich um Herrn Julian Gaisbauer 
    (BSc), der die Rolle des Betreuers im Praktikum übernommen hat. Dieser war mit der Einarbeitung in die Projekte und die Unterstützung bei der Arbeit betraut. Außerdem stand er für 
    Fragen und Probleme aller Art zur Verfügung. Die zweite Person war Herr Theodor Hartleitner (MSc), der einen Großteil der Aufgaben und Anforderungen definierte. Bei spezifischeren 
    Fragen zu den Anforderungen und der Software stand er ebenfalls zur Verfügung. Mit den anderen Teammitgliedern wurde wenig Zusammenarbeit geleistet, da diese meist an anderen Projekten arbeiteten.


\section{Übersicht über die Projekte}
    Während der Zeit des Praktikums wurde an zwei Projekten gearbeitet. Beide Projekte sind im Bereich des \gls{oepnv} angesiedelt und sollten somit Verkehrsunternehmen die Verwaltung von ihrer Daten 
    vereinfachen. Das Projekt \emph{"ITCS-Management"} sollte dabei die Visualisierung und Verwaltung von daten wie Fahrpläne und Fahrzeugen ermöglichen. Das zweite Projekt 
    \emph{"ITCS"} sollte eine Schnittstelle zu anderen Systemen bieten sodass Daten aus anderen Systemen importiert und exportiert werden können. Weiters fungierte es als Bibliothek 
    die von anderen Projekten verwendet werden kann um beispielsweise Datenvalidierung zu ermöglichen oder Datenbankobjekte automatisch zu generieren.

\subsection{Das Projekt \emph{"ITCS-Management"}}
    Das Projekt \emph{"ITCS-Management"} ist eine Webanwendung, die es ermöglicht, Daten wie Fahrpläne, Fahrzeuge und Haltestellen zu verwalten. Es wurde mit dem Ziel entwickelt, den 
    Verwaltungsaufwand für Verkehrsunternehmen zu reduzieren und eine benutzerfreundliche Oberfläche zu bieten. Es wurde mit C\# und Blazor umgesetzt. Bei der Datenbank
    handelt es sich um eine SQL-Server-Datenbank, die ein VDV-452~\cite{VDV452} konformes
    Datenmodell österreichischer Verkehrsunternehmen abbildet. Gelegentlich musste dieses
    Schema auch erweitert oder angepasst werden. Der Zugriff auf die Datenbank erfolgte durch das \gls{efc}. 
    Das Projekt wurde nicht neu begonnen sondern sondern um neue Seiten und Funktionen erweitert.
    Dabei handelte es eine Seite für die Verwaltung von 
