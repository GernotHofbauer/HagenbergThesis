\chapter{Implementierung}\label{ch:Implementierung}
\section{Implementierung "ITCS-Management"}\label{sec:ImplementierungITCS-Management}
    Die Benutzeroberflächen der \emph{ITCS-Management}-Anwendung wurden wie bereits erwähnnt mit Blazor implementiert. Dabei werden pro Seite 
    zwei Dateien angelegt. Eine \texttt{.razor}-Datei, die das Layout der Seite über eine XML-ähnliche Syntax definiert
    und eine \texttt{razor.cs}-Datei, die als \texttt{partial} markiert werden muss und alle benötigten Methoden und Logik hinter der Seite enthält. Manchmal ist die Verwendung einer \texttt{razor.css}-Datei 
    auch sinnvoll, um die Seite zu stylen. Die Logic und die Css-klassen könnten zwar auch inder der \texttt{.razor}-Datei definiert werden,
    aber das würde die Übersichtlichkeit stark verringern. 
    
    Der Datenbankzugriff erfolgt über mehrere Schichten. \emph{Repositories} sind dafür zuständig, 
    die Daten aus der Datenbank zu laden und zu speichern. Sie führen normalerweise nur grundlegende Interaktionen mit der Datenbank über den Datenbank-Kontext des \gls{efc} aus.
    Diese \emph{Repositories} werden dann von den \emph{Services} verwendet, die die Logik der Anwendung enthalten. Ein großer 
    Vorteil der Verwendung von Blazor im Vergleich zu anderen Frontend-Technologien wie Angular oder React ist, dass die gesamte Logik der Anwendung in
    C\# geschrieben werden kann. Dadurch werden sind keine HTTP-Endpoints benötigt da auf die Daten des Teilprojekts für den Datenbankzugriff direkt zugegriffen werden kann.
    Die Services und Repositories wurden dann in der \texttt{Program.cs}-Datei der Anwendung für die Dependency-Injection registriert,
    damit sie in den Blazor-Komponenten verwendet werden können.

    Für die Darstellung der Daten im Frontend wurde auf UI-Komponentenbibliothek \emph{Radzen Blazor}~\cite{radzen} zurückgegriffen. 
    Diese Bibliothek bietet eine Vielzahl an vorgefertigten Komponenten, die das Erstellen von Benutzeroberflächen erleichtern.
    
    \begin{program} [H]
        \caption{Ermittlung der zu generierenden Fahrten}
        \label{prog:RideGeneration}
    \begin{CsCode}[numbers=left]
RideViewItem pRide = null;
foreach (RideViewItem cRide in sortedRides)
{
    if (pRide is null){ pRide = cRide; continue; }

    int previousEndTime = pRide.StartTimeInSeconds + pRide.DriveTimeInSeconds;
    int timeBetweenRides = cRide.StartTimeInSeconds - previousEndTime;

    if (!IsSpecialStopPoint(pRide.DestinationLocationTypeNr) &&
        !IsSpecialStopPoint(cRide.StartLocationTypeNr))
    {
        bool isSameLocation = AreLocationsEqual(
            pRide.DestinationLocationNr, pRide.DestinationLocationTypeNr,
            cRide.StartLocationNr, cRide.StartLocationTypeNr);

        if (!isSameLocation && this.generateNonProductiveRides)
        {
            RideViewItem existingEmptyRide = existingGeneratedRides
                .FirstOrDefault(r => r.RouteNr == 9999 &&
                    r.StartLocationNr == pRide.DestinationLocationNr &&
                    r.StartLocationTypeNr == pRide.DestinationLocationTypeNr &&
                    r.DestinationLocationNr == cRide.StartLocationNr &&
                    r.DestinationLocationTypeNr == cRide.StartLocationTypeNr);

            if (existingEmptyRide is not null)
            {
                neededRides.Add(existingEmptyRide.RideNr);
                int newStartTime = pRide.StartTimeInSeconds + 
                    pRide.DriveTimeInSeconds + pRide.RideStopTimeInSeconds;
                if (existingEmptyRide.StartTimeInSeconds != newStartTime)
                    await this.UpdateRideStartTime(existingEmptyRide, newStartTime);
            }
            else
                await this.GenerateForEmptyRide(pRide, cRide, generator, trip);
        }
        else if (isSameLocation && timeBetweenRides > minimumBreakTime && 
                    this.generateBreaks &&
                    IsNoSpecialRide(pRide) && !IsConnectThrough(pRide, cRide))
        {
            RideViewItem existingBreak = existingGeneratedRides
                .FirstOrDefault(r => r.RouteNr == 9989 &&
                    r.StartLocationNr == pRide.DestinationLocationNr &&
                    r.StartLocationTypeNr == pRide.DestinationLocationTypeNr);

            if (existingBreak is not null)
            {
                neededRides.Add(existingBreak.RideNr);
                int newStartTime = pRide.StartTimeInSeconds + 
                    pRide.DriveTimeInSeconds + pRide.RideStopTimeInSeconds;
                if (existingBreak.StartTimeInSeconds != newStartTime)
                    await this.UpdateRideStartTime(existingBreak, newStartTime);
            }
            else
                await this.GenerateBreak(pRide, trip, generator);
        }
    }
}\end{CsCode} \end{program}

    In Programm~\ref{prog:RideGeneration} ist ein Ausschnitt der Methode zu sehen, die für die Ermittlung der zu generierenden Fahrten zuständig ist. Um genauer zu sein,
    handelt es sich um den Teil der die zu generierenden Leerfahrten und Pausen ermittelt. Zunächst werden dafür allen Fahrten in dem Umlauf nach ihrer Startzeit sortiert.
    Anschließend wird in einer Schleife über alle sortierten Fahrten iteriert wobei jede Fahrt gemeinsam mit der vorherigen betrachtet wird. Sind der Zielpunkt der vorherigen Fahrt
    und der Startpunkt der aktuellen Fahrt nicht gleich, wird eine Leerfahrt benötigt. Um unnötige Generierungen zu vermeiden, wird zunächst geprüft, ob eine Leerfahrt bereits existiert.
    Sollte dies der Fall sein, wird diese Fahrt in die Liste der benötigten Fahrten aufgenommen und die Startzeit der Leerfahrt aktualisiert, falls sie sich geändert hat.
    Sind der Zielpunkt der vorherigen Fahrt und der Startpunkt der aktuellen Fahrt gleich, wird geprüft, ob eine Pause generiert werden soll. Solche Sonderfahrten werden nur generiert, wenn
    der Kunde dies explizit in den Einstellungen aktiviert hat und wenn es sich bei beiden Fahrten nicht um Sonderfahrten handelt. Sollte es um Anschlussfahrten gehen,
    werden ebenfalls keine Pausen generiert.

\section{Implementierung "ITCS"}\label{sec:ImplementierungITCS}
    
    Das Projekt \emph{ITCS} wurde um die Klasse \texttt{EntityGenerator} erweitert. Diese Klasse wird mit einem Datenbank-Connetion-String initialisiert 
    und kann dann verwendet werden, um Daten in die Datenbank zu generieren. Für manche Operationen wurde auf den bereits erwähnten Datenbank-Manager zurückgegriffen. 
    Die Transaktionsverwaltung wurde aber für jede Methode selbst implementiert, da der Datenbank-Manager nicht für alle Operationen geeignet ist. Oftmals 
    wurden Operationen direkt als SQL-Statements ausgeführt. 
    Die Methoden von \texttt{EntityGenerator} sollten als Bibliothek in anderen Projekten verwendet werden können.
    \begin{program} [H]
        \caption{Ermittlung der zu generierenden Fahrten}
        \label{prog:RideGeneration}
    \begin{CsCode}[numbers=left]
public virtual int GenerateCourseNr(string ccid, int routeNr)
{
    if (string.IsNullOrEmpty(ccid))
        throw new ArgumentException("ControlCenterId darf nicht leer sein.",
             nameof(ccid));

    using ITCSDatabaseManager itcsDbManager = new(this.connectionString, 
        this.commandTimeout);
    var config = itcsDbManager.GetControlCenterConfiguration(ccid) ??
        throw new ArgumentException(
            $"Keine Konfiguration für '{ccid}' gefunden.", nameof(ccid));

    int courseNoLength = config.CourseNoLength > 0 ? config.CourseNoLength : 3;

    int maxCourseValue = 899;
    if (courseNoLength > 0)
        maxCourseValue = (int)Math.Pow(10, courseNoLength) - 101;

    List<VersionValidity> versions = 
        itcsDbManager.GetCurrentAndFutureVersions(ccid);
    if (versions.Count == 0)
        throw new InvalidOperationException(
            "Es wurden keine aktuellen oder zukünftigen Versionen gefunden.");

    int highestCourseNr = 0;
    int currentPrefix = routeNr;
    bool found = false;
    while (!found && currentPrefix >= routeNr - 10)
    {
        int internalHighestCourseNr = 0;
        foreach (VersionValidity version in versions)
        {
            IList<Ride> rides = itcsDbManager.GetRidesByRoute(ccid,
                version.VersionNr, currentPrefix, null);

            foreach (Ride ride in rides)
                if (ride.CourseNr > highestCourseNr && 
                    ride.CourseNr % currentPrefix <= maxCourseValue)
                    internalHighestCourseNr = ride.CourseNr;
        }
        if (internalHighestCourseNr % currentPrefix < maxCourseValue)
        {
            highestCourseNr = internalHighestCourseNr;
            found = true;
            break;
        }
        currentPrefix--;
    }
    if (!found)
        throw new InvalidOperationException("keine gültige Kursnummer gefunden.");

    return (currentPrefix * (int)Math.Pow(10, courseNoLength)) + 
        ((highestCourseNr + 1) % currentPrefix);
}\end{CsCode} \end{program}