\chapter{Kurzfassung}% Umfang der Kurzfassung: ca.\ 200 Worte.
Dieser Praktikumsbericht beschreibt die Arbeit des Praktikanten im Rahmen seines Praktikums bei der Firma ITCS GmbH. Der Schwerpunkt lag auf der Entwicklung
von Software für das ITCS-Management-System, das in der öffentlichen Verkehrsinfrastruktur eingesetzt wird. Das Herzstück dieser Software ist eine Webanwendung,
die es ermöglicht, Umläufe aus einzelnen Fahrten zu erstellen. Dabei sollten verschiedene benötigte Daten auf der darunterliegenden Datenbank generiert werden.
Außerdem sollten diese Umläufe auch automatisch auf Durchführbarkeit geprüft werden.
In diesem Praktikum wurde mit der Programmiersprache \emph{C\#} und dem \emph{.NET}-Framework gearbeitet. Der Datenbankzugriff erfolgte über die \emph{Entity Framework Core}-Bibliothek, 
die eine objektorientierte Abstraktionsebene für den Datenbankzugriff bereitstellt. Das Frontend wurde mit \emph{Blazor} entwickelt, einer modernen Webtechnologie, 
die es ermöglicht, interaktive Webanwendungen mit C\# zu erstellen. Durch die Verwendung von \emph{Blazor} konnte eine nahtlose Integration zwischen Frontend und Backend erreicht werden,
was die Entwicklung effizienter und flexibler machte.
Während der Implementierung diese Umlaufeditors musste auch teilweise das bestehende Datenmodell angepasst werden, um die neuen Anforderungen zu erfüllen. 
Da die Datenbank mit Daten aus anderen Systemen gefüllt wird, musste auch ein anderes, bestehendes Projekt, das für den Import und Export von Daten benutzt ist, angepasst werden.
Dabei musste darauf geachtet werden, dass die Daten weiterhin mit dem \emph{VDV-452}-Standard kompatibel bleiben, um die Interoperabilität mit anderen Systemen zu gewährleisten.