\chapter{Zusammenfassung und Erfahrungen}\label{ch:ZusammenfassungundErfahrungen}
    Da der genaue Umfang der geplanten Arbeit des Praktikums von der Firma nie genau festgelegt oder kommuniziert wurde,
    ist es schwierig zu festzustellen, ob die Arbeit nach den Erwartungen der Firma verlaufen ist. Es wurden nicht alle Aufgaben am Taskboard abgeschlossen. 
    Viele Aufgaben wurden jedoch während der aktiven Arbeit an den Anwendungen neu angelegt, wobei auch klar erkennbar war, dass
    diese Aufgaben nicht im ursprünglichen Plan enthalten waren. Das Feedback der Betreuer war auch durchwegs positiv. Alle wichtigsten Features des Herzstückes 
    der Arbeit, der \emph{Umlaufeditor}, wurden implementiert und funktionieren wie geplant.

    Es müssen jedoch noch einige Integrationstests durchgeführt werden, um die Qualität der Software zu gewährleisten. Auch sollten manche Dinge von der 
    Firma noch genauer überdacht bzw. überarbeitet werden. Beispielsweise returniert der Dienst zur Ermittlung der Strecken für
    Sonderfahrten keinen spezifischen Fehlercode, sondern nur \texttt{500 Internal Server Error}. Da während des Praktikums kein Zugriff keine Einsicht in den Quellcode
    des Dienstes möglich war, war es nicht möglich genau zu ermitteln, warum manche Fehler bei der Generierung der Strecken auftreten.

    Durch die Arbeit in der Firma konnte ich viele neue Erfahrungen sammeln. Ich erlangte wichtige Einblicke in die Arbeit eines agilen Softwareentwicklungsteams.
    Dazu kann auch gehören, dass neue Anforderungen während der Entwicklung definiert werden oder bestehende Anforderungen stark verändert werden.
    Auch im Umgang mit verschiedenen Technologien konnte ich mein Wissen erweitern. So habe ich zum Beispiel sehr viel über die Arbeit mit Blazor gelernt. An der FH 
    wurde Blazor nämlich nur kurz in einer Vorlesung besprochen, ohne viel praktische Erfahrung damit zu sammeln. Auch die starke Verwendung von \gls{efc} konnte mir 
    helfen, meine Kenntnisse in diesem Bereich zu erweitern. 
    Man kann zusammenfassend sagen, dass ich durch dieses Praktikum sowohl meine technischen Fähigkeiten als 
    Softwareentwickler erweitern konnte als auch Erfahrungen über die Arbeit in einem Entwicklerteam sammeln konnte. Diese wichtigen Einblicke in die Arbeitswelt
    der Softwareentwicklung werden mir in meiner zukünftigen beruflichen Karriere sicherlich von großem Nutzen sein.

