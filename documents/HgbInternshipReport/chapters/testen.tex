\chapter{Testen}\label{chap:testen}
    Das Testen des im Rahmen des Praktikums erstellten Codes war nicht Schwerpunkt der Arbeit des Praktikanten. Meistens wurden deswegen vor allem im Projekt \emph{"ITCS-Management"} nur 
    manuelle Integrationstests über das UI durchgeführt. Bei Bedarf wurden auch Sql-Statements direkt in der Datenbank ausgeführt, um die Korrektheit der Daten zu überprüfen. 
    Weitere Tests wurden von anderen  Mitarbeiter der Firma durchgeführt ohne direkte Beteiligung des Praktikanten. Sollten 
    Fehler aufgetreten sein, wurden diese in einem Bug-Tracker dokumentiert und vom Praktikanten später behoben.

    Bei zwei verschiedenen Aufgaben im Projek \emph{"ITCS"} wurden jedoch explizit automatisierte Tests geachtet durch den Praktikanten gefordert.
    Diese wurden mit der \emph{xUnit}-Bibliothek~\cite{xunit}, einer weit verbreiteten Testbibliothek für .NET-Anwendungen erstellt.
    Für das Mocken von Datenbankzugriffen wurde die \emph{Moq}-Bibliothek~\cite{moq} verwendet. Diese ermöglicht es, Objekte zu erstellen, die das Verhalten von Abhängigkeiten simulieren.
    Einerseits wurde dabei Unit-Tests für eine Methode erstellt die für Fahrten neue \texttt{CourseNr}-Werte generiert. Diese Werte sind zwar nicht Teil des Primärschlüssels, sind jedoch 
    für weitere Verwendung der Daten auf anderen Systemen wichtig und müssen aus diesem Grund eindeutig sein und eine bestimmte Struktur aufweisen. 

    \begin{program} [H]
        \caption{Beispiel eines Unit-Tests für die Generierung von CourseNr-Werten}
        \label{prog:UnitTest}
    \begin{CsCode}[numbers=left]
[Fact]
public void GenerateCourseNrIgnoresOutOfRangeValuesWhenGeneratingNextNumber()
{
    Mock<ITCSDatabaseManager> dbMock = new("conn", 30) { CallBase = true };
    dbMock.Setup(m => m.GetControlCenterConfiguration("cc1"))
        .Returns(new ControlCenterConfiguration { CourseNoLength = 3 });
    dbMock.Setup(m => m.GetCurrentAndFutureVersions("cc1"))
        .Returns(new List<VersionValidity> { new() { VersionNr = 1 } });

    dbMock.Setup(m => m.GetRidesByRoute("cc1", 1, 9999, null))
        .Returns(new List<Ride> {
            new() { CourseNr = 9999005 },
            new() { CourseNr = 9999950 },
            new() { CourseNr = 9999003 }
        });

    TestEntityGenerator generator = new(dbMock.Object);
    int result = generator.PublicGenerateCourseNr("cc1", 9999);
    Assert.Equal(result, Is.EqualTo(9999006));
}\end{CsCode}
    \end{program}

    In Programm~\ref{prog:UnitTest} ist ein Beispiel eines Unit-Tests für die Generierung von CourseNr-Werten zu sehen. 
    Anfangs wird ein Mock-Objekt für die Datenbank erstellt, das die Abhängigkeit von der echten Datenbank simuliert. Dabei wird für 
    einzelne Methodenaufrufe das erwartete Verhalten festgelegt. Anschließend wird die zu testende Methode aufgerufen und das Ergebnis überprüft.

    Die weiteren mit Tests versehen Methoden, generieren oder verändern Daten in der Datenbank, sollten bestimmte Objekte wie Betriebshöfe Sonderhaltepunkte in die
    Datenbank eingefügt oder modifiziert werden. Da sie für den Datenbankzugriff auf sehr viele eigens dafür erstellte Sql-Statements zurückgreifen, wurden sie in Integrationstests 
    mit direkten Abhängigkeiten zu einer Test-Datenbank getestet. Benutzt wurde dafür eine Firmeneigene Test-Datenbank, die regelmäßig mit den aktuellen Daten der Produktionsdatenbank gefüllt wird.
    Die Tests wurden aber dennoch so gestalten, dass sie nur an das Datenschema gebunden sind und nicht an die konkreten Daten.
    Es wurde ebenfalls \emph{xUnit} verwendet und die Tests ähneln dem Beispiel in Programm \ref{prog:UnitTest}, nur dass hier die Datenbankabfragen nicht gemockt, sondern 
    direkt ausgeführt werden und alle erstellten Objekte abgespeichert und anschließend wieder gelöscht werden, um die Test-Datenbank nicht zu verändern.