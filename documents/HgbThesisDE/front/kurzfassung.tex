\chapter{Kurzfassung}

    Diese Bachelorarbeit befasst sich mit einigen Neuerungen von \emph{Project Loom} für Java. Eingegangen wird dabei auf \emph{virtuelle Threads}, \emph{StructuredTaskScopes},
    die auch auf virtuellen Threads basieren
    und \emph{ScopedValues}.
    Zu jeder dieser Technologien wird zunächst eine technische Übersicht geboten, bei der darauf eingegangen wird, inwiefern sie sich von den bereits bekannten Technologien unterscheiden.
    Darauf folgt eine eine Erklärung wie sie in Java verwendet werden können, wobei einige Beispiele gezeigt werden. 
    Bei \texttt{StructuredTaskScopes} beinhaltet dies auch das Einbinden eigener Logik und Fehlerbehandlung durch Ableitung von der Basisklasse. Bei \texttt{ScopedValues} wird auch auf die Unterschiede zu \texttt{ThreadLocal} eingegangen. 
    Um eine besseren Vergleich ziehen zu können, wird auch auf die Performance eingegangen. Dazu wurden sechs einfache Benchmarks,
    welche das Laufzeitverhalten untersuchen, erstellt und durchgeführt. Dabei kam das
    „Java Microbenchmark Harness (JMH)“ zum Einsatz. Aufgrund Schwierigkeiten
    beim Messen des Speicherverbrauchs wurde zu diesem Aspekt nur ein Test durchgeführt. Jedes Messergebnis ist mit eine Analyse versehen.
    Auf Basis der Recherche und der Ergebnisse der Benchmarks wird letzten Endes im Fazit eine Zusammenfassung der Stärken und potentiellen Anwendungsszenarien gegeben.