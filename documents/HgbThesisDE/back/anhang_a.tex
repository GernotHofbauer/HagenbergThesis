\chapter{Speichermessung}
\label{app:Speichermessung}
    Dieser Anhang beinhaltet den Quelltext für Kapitel \ref{sec:Speicherbedarf}.
    \begin{program} [H]
        \caption{Speicherbedarfsmessung}
        \label{prog:Speicherverbrauch}
    \begin{JavaCode}[language=Java, numbers=left]
public class MemoryUsageComparison {

    private static final int ARRAY_SIZE = 1024 * 1024;
    // Large array to simulate memory usage

    private static final InheritableThreadLocal<byte[]> threadLocal = 
        new InheritableThreadLocal<>() {
            @Override
            protected byte[] initialValue() {
                return new byte[ARRAY_SIZE];
            }
        };

    private static final ScopedValue<byte[]> scopedValue = 
        ScopedValue.newInstance();

    public static void main(String[] args) throws Exception {
        // Warmup the jvm
        System.out.println("Warming up JVM...");
        for (int i = 0; i < 5; i++) {
            measureThreadLocal(10);
            measureScopedValue(10);
            System.gc();
            Thread.sleep(100);
        }

        int threadCount = 1000;
        int measurements = 5;
        long[] threadLocalMemory = new long[measurements];
        long[] scopedValueMemory = new long[measurements];
    \end{JavaCode}
\end{program}

\begin{program} [H]
    \begin{JavaCode}[language=Java, firstnumber=last]
        System.out.println("\nStarting measurements...");
        for (int i = 0; i < measurements; i++) {
            System.gc();
            Thread.sleep(1000);
            long baselineMemory = getUsedMemory();
            System.out.println();
            System.out.println(STR."Measurement \{i + 1}");
            System.out.println("Baseline Memory: " + fMem(baselineMemory));
            // ThreadLocal test
            long beforeThreadLocal = getUsedMemory();
            measureThreadLocal(threadCount);
            long afterThreadLocal = getUsedMemory();
            threadLocalMemory[i] = afterThreadLocal - beforeThreadLocal;
            System.out.println("ThreadLocal Memory Usage: " 
                + fMem(threadLocalMemory[i]));

            System.gc();
            Thread.sleep(1000);

            // ScopedValue test
            long beforeScoped = getUsedMemory();
            measureScopedValue(threadCount);
            long afterScoped = getUsedMemory();
            scopedValueMemory[i] = afterScoped - beforeScoped;
            System.out.println("ScopedValue Memory Usage: " 
                + fMem(scopedValueMemory[i]));
        }

        System.out.println("\nFinal Results:");
        System.out.println("avg ThreadLocal Memory Usage: " + 
            fMem(avg(threadLocalMemory))});
        System.out.println("avg ScopedValue Memory Usage: " + 
            fMem(avg(scopedValueMemory))});
        System.out.println("Memory Difference: " + 
            fMem(Math.abs(avg(threadLocalMemory) - avg(scopedValueMemory)))});
    }

    private static void measureThreadLocal(int threadCount) throws Exception {
        try (var scope = new StructuredTaskScope<Void>()) {
            for (int i = 0; i < threadCount; i++) {
                scope.fork(() -> {
                    byte[] data = threadLocal.get();
                    // Play around with the data
                    data[0] = (byte) System.nanoTime();
                    Thread.sleep(50);
                    return null;
                });
            }
            scope.join();
        }
    }

\end{JavaCode}
\end{program}

\begin{program} [H]
    \begin{JavaCode}[language=Java, firstnumber=last]
    private static void measureScopedValue(int threadCount) throws Exception {
        ScopedValue.where(scopedValue, new byte[ARRAY_SIZE]).run(() -> {
            try (var scope = new StructuredTaskScope<Void>()) {
                for (int i = 0; i < threadCount; i++) {
                    scope.fork(() -> {
                        byte[] data = scopedValue.get();
                        // Play around with the data
                        data[0] = (byte) System.nanoTime();
                        Thread.sleep(50);
                        return null;
                    });
                }
                scope.join();
            } catch (InterruptedException e) {
                throw new RuntimeException(e);
            }
        });
    }
    
    private static long getUsedMemory() {
        MemoryMXBean memoryBean = ManagementFactory.getMemoryMXBean();
        return memoryBean.getHeapMemoryUsage().getUsed() +
                memoryBean.getNonHeapMemoryUsage().getUsed();
    }

    private static String fMem(long bytes) {...}

    private static long avg(long[] values) {...}
}\end{JavaCode}
    \end{program}
