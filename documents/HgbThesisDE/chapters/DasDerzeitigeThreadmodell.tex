\chapter{Das derzeitige Threadmodell}
\label{cha:DasDerzeitigeThreadmodell}                   % 5 Seiten
    Java und die \gls{jvm} stellen mit der Klasse\texttt{Thread} und dem Interface \texttt{Runnable} im Paket \texttt{java.lang} grundlegende Mechanismen zur Thread-Verarbeitung bereit.
    Mit dem Start der JVM werden automatisch der Haupt-Thread und administrative Threads wie der Garbage Collector gestartet. Jeder Thread kann seinerseits weitere Threads instanziieren \cite{GrundKursBS}.
    
\section{Funktionsweise}                                         
\label{sec:Funktionsweise}

    Bisher war jede jeder Thread in der \gls{jvm} eine Instanz von \texttt{java.lang.Thread} auch \gls{pt} genannt. Diese Klasse stellt einen Wrapper-Klasse um \Glspl{ot} dar und Instanzen dieser Klasse
    binden den darunterliegenden \gls{ot}  während ihrer gesamten Lebenszeit an sich. Dadurch ist die Anzahl der gesamt verfügbaren \Glspl{pt} durch die Anzahl der \Glspl{ot} begrenzt und die Verwaltung
    der Ressourcen wie CPU-Rechenzeit unterliegt dem Betriebssystem. \cite{JEP425}
    \texttt{java.lang.Thread} implementiert das Interface \texttt{Runnable} und lässt Ableitungen zu. Die \gls{jvm} kann bei Verwendung eines \texttt{CachedThreadPool}s Threads die ihre Arbeit bereits 
    abgeschlossen haben wiederverwenden um eine neue Instanziierung zu vermeiden \cite{Executors}.

    
\section{Probleme}                                         
\label{sec:Probleme}

    \Glspl{ot} bringen viele Probleme mit sich. Eines der größten ist deren Ressourcenintensität.
    Sie müssen alle verschiedene Sprachen und Aufgaben unterstützen. Es ist wichtig, dass sie die Ausführung eines Prozesses unterbrechen und
    wiederaufnehmen können, 
    was das Speichern des Zustands (z.B. des Stack-Pointers und des Program-Counters) erfordert. Da das Betriebssystem keine Informationen darüber hat, wie
    eine Sprache ihren Stapel verwaltet, wird der Speicher großzügig zugeteilt und kann zur Laufzeit weiter wachsen, aber nicht dynamisch schrumpfen,
    was bei Aufgaben mit einer langen durchgängigen Ausführung zu Problemen führen kann. Da der Betriebssystem-Kernel jegliche Art von Threads verwalten muss,
    die unterschiedlichste Aufgaben ausführen können, von HTTP-Requests bis zu leistungsintensiven Berechnungen, müssen bei der
    Verwaltung der CPU Kompromisse eingegangen werden. Dies führt dazu, dass einem Thread CPU-Zeit zugeteilt wird, obwohl dieser nur wartet. \cite{ProjectLoom}
    

