\chapter{Das derzeitige Threadmodell}
\label{cha:DasDerzeitigeThreadmodell}                   % 5 Seiten
    Java und die \gls{jvm} stellen mit der Klasse \texttt{Thread} und dem Interface \texttt{Runnable} im Paket \texttt{java.lang} grundlegende Mechanismen zur Thread-Verarbei\-tung bereit.
    Mit dem Start der JVM werden automatisch der Haupt-Thread und administrative Threads wie der Garbage Collector gestartet. Jeder Thread kann seinerseits weitere Threads instanziieren \cite{GrundKursBS}.
  
\section{Funktionsweise}                                         
\label{sec:Funktionsweise}

    Bisher war jeder Thread in der \gls{jvm} eine Instanz von \texttt{java.lang.Thread} auch \gls{pt} genannt. Diese Klasse stellt einen Wrapper-Klasse um \Glspl{ot} dar und Instanzen dieser Klasse
    binden den darunterliegenden \gls{ot}  während ihrer gesamten Lebenszeit an sich. Dadurch ist die Anzahl der gesamt verfügbaren \Glspl{pt} durch die Anzahl der \Glspl{ot} begrenzt und die Verwaltung
    der Ressourcen wie CPU-Rechenzeit unterliegt dem Betriebssystem \cite{JEP425}.
    \texttt{java.lang.Thread} implementiert die Schnittstelle \texttt{Runnable} und lässt Ableitungen zu.
    Außerdem können sogenannte \emph{Thread-Pools} genutzt werden. Dabei handelt es sich um einen Mechanismus der mehrere \texttt{worker-threads} verwaltet. Diese existieren unabhängig von dem \texttt{Runnable},
    das sie ausführen. Die Verwendung von Thread-Pools verringert den Overhead der sich durch die Erstellung von Threads ergibt \cite{ThreadPool}. Thread-Pools können über die Konstruktoren der Klasse \texttt{ThreadPoolExecutor}
    erstellt werden, es ist jedoch üblich auf die Factory-Methoden der Klasse \texttt{Executors} zurückzugreifen. Die Methode \texttt{newFixedThreadPool(int int nThreads)} ermöglicht es 
    die Menge an Threads konstant zu halten. Wird ein Thread beendet, durch einen Fehler oder Abschluss der Arbeit, erstellt der Pool sofort einen neuen. Andere Implementierungen wie der 
    \texttt{newThread\-PerTaskExecutor} versuchen für jede Aufgabe einen eigenen Thread zu erstellen. Alle Aufgaben die nicht direkt bearbeitet werden können, 
    speichert der Thread-Pool in einem Behälter zwischen. Thread-Pools müssen nach der Verwendung wieder geschlossen werden, entweder mittels \texttt{shutdown()} oder der Verwendung eines
    try-with-resources-Blocks.
    Die \gls{jvm} kann bei Verwendung eines \texttt{CachedThreadPool}s Threads, die ihre Arbeit bereits 
    abgeschlossen haben, wiederverwenden, um eine neue Instanziierung zu vermeiden. Dabei werden Threads, die ihre Aufgabe bereits abgeschlossen haben, für weitere 60 Sekunden am Leben erhalten um 
    neue Aufgaben darauf ausführen zu können. Erst danach werden sie freigegeben. Natürlich können bei Bedarf neue Threads erstellt werden \cite{Executors}.

    
\section{Probleme}                                         
\label{sec:Probleme}

    \Glspl{ot} bringen viele Probleme mit sich. Eines der größten ist deren Ressourcenintensität.
    Sie müssen alle verschiedene Sprachen und Aufgaben unterstützen. Es ist wichtig, dass sie die Ausführung eines Prozesses unterbrechen und
    wiederaufnehmen können, 
    was das Speichern des Zustands (z.B. des Stack-Pointers und des Program-Counters) erfordert. Da das Betriebssystem keine Informationen darüber hat, wie
    eine Sprache ihren Stapel verwaltet, wird der Speicher großzügig zugeteilt und kann zur Laufzeit weiter wachsen, aber nicht dynamisch schrumpfen,
    was bei Aufgaben mit einer langen durchgängigen Ausführung zu Problemen führen kann. Da der Betriebssystem-Kernel jegliche Art von Threads verwalten muss,
    die unterschiedlichste Aufgaben ausführen können -- von HTTP-Requests bis zu leistungsintensiven Berechnungen -- müssen bei der
    Verwaltung der CPU Kompromisse eingegangen werden. Dies führt dazu, dass einem Thread CPU-Zeit zugeteilt werden kann, obwohl dieser nur wartet. Außerdem wird die Menge 
    an verfügbaren Threads durch das Betriebssystem begrenzt. So kann der Durchsatz einer "Thread pro Aufgabe" - Architektur eingeschränkt sein, auch wenn das Limit der 
    Hardware noch nicht erreicht wurde \cite{ProjectLoom}. Bei Verwendung von Thread-Pools wird schnell klar, dass diese sehr unstrukturiertes Vorgehen ermöglichen. Ein Thread kann beispielsweise 
    einen Thread-Pool erstellen und ein anderer erstellt dafür Aufgaben und ein dritter Thread erhält irgendwie eine Referenz auf das \texttt{Future}-Objekt und kann somit auf das Ergebnis warten.
    Es besteht also eine lose Bindung zwischen den Eltern- und Kind-Threads. In manchen Fällen mag sich dies zwar als vorteilhaft erweisen, jedoch steigt dadurch auch der Aufwand einer potentiellen 
    Fehlersuche. In vielen Fällen ist eine stärkere Kontrolle durch den Eltern-Thread erwünscht um \zB eine zentrale Fehlerbehandlung zu ermöglichen oder die Freigabe von Ressourcen zu garantieren.
    Das Einbinden allgemeiner Logik, die für alle Threads des Pools gelten soll erweist sich auch als sehr schwierig. Sollten beispielsweise alle Aufgaben eines Thread-Pools abgebrochen werden,
    wenn auch nur eine fehlschlägt, lässt sich dies derzeit oft nur schwer umsetzen und das Ergebnis neigt meist dazu sehr unübersichtlich zu werden \cite{JEP453}. 
    

