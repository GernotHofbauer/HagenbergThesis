\chapter{Einleitung}
\label{cha:Einleitung}

    Seit einigen Jahren Arbeitet Oracle an Möglichkeiten zur Verbesserung der Skalierbarkeit von Java-Anwendungen im Projekt "Loom".
    Der Hauptansatz dieses Vorhabens ist die Einführung von virtuellen Threads, die von der JVM verwaltet werden. 
    Diese Threads sind leichtgewichtiger als klassische Plattform Threads und können in größerer Anzahl erzeugt werden.
    So können bestimmte Anwendungen mit hoher Nebenläufigkeit zukünftig effizienter gestaltet werden. Diese Bachelorarbeit 
    beleuchtet diese Technologien und stellt die Neuerungen dem bereits Bekanntem gegenüber.

\section{Ziel der Arbeit}
\label{sec:Ziel}

    Das Ziel dieser Bachelorarbeit ist es, die grundlegenden Eigenschaften und Limitierungen des bestehenden Thread-Konzepts zu untersuchen und die Motivationen für die Neuerungen
    durch Projekt Loom zu verstehen. Ein Überblick über Projekt Loom und die damit verbundenen Technologien wird gegeben, wobei der Schwerpunkt auf Virtual Threads liegt. 
    Die Arbeit soll zeigen, wie und in welchen Fällen diese Neuerungen in der Praxis angewendet werden können und welche neuen Möglichkeiten sich dadurch ergeben. 
    Es wird auch verdeutlicht, welche bestehenden Probleme der parallelen Ausführung durch die neuen Technologien nicht gelöst werden können. 
    Durch Benchmarks sollen Laufzeit und Speicherverbrauch unter verschiedenen Umständen analysiert werden, um daraus Schlüsse zu ziehen, 
    in welchen Fällen die neuen Technologien verwendet werden sollten und in welchen nicht. Als konkretes Ergebnis wird eine Sammlung kleinerer Programme erstellt, 
    die die neuen Technologien in Projekt Loom demonstrieren und die Unterschiede zu den bisherigen Technologien aufzeigen.



