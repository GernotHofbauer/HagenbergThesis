\chapter{Messungen}
\label{cha:Messungen}
    In Rahmen dieser Arbeit wurden verschiedene Messungen und Benchmarks der Laufzeit durchgeführt um Vergleiche zwischen den bereits bekannten Technologien und den Neuerungen in Projekt Loom 
    durchführen zu können. 

\section{Testumgebung}                                         
\label{sec:Testumgebung}

    Alle getesteten Programmen wurden mittels Java Development Kit (JDK) \emph{Eclipse Temurin 22.0.2} kompiliert und ausgeführt. Als JVM-Optionenen wurde 
\begin{verbatim}
--add-exports java.management/sun.management=ALL-UNNAMED 
--add-exports jdk.management/com.sun.management.internal=ALL-UNNAMED 
-XX:+UnlockDiagnosticVMOptions -XX:+DebugNonSafepoints\end{verbatim}
    benutzt.
    Das System nutzt Windows 11 Professional als Betriebssystem und ist mit einem "AMD Ryzen 7 7700x" Prozessor ausgestattet. Dabei handelt es sich um eine CPU mit x86-64-Anweisungssatz und 8 
    Kernen mit einer Taktfrequenz zwischen 4,5 und 5,4 GHz. Als Arbeitsspeicher kamen  32 GB DDR5-RAM zum Einsatz die auf 6400MHz takten. Die integrierte Graphikeinheit der CPU wurde der CPU wurde 
    ,um eine zusätzliche Wärmebelastung zu vermeiden, deaktiviert. 
    Für die Laufzeitmessungen wurde das "Java Microbenchmark Harness (JMH)" Version 1.37 verwendet. 


\section{Laufzeitmessungen bei Threads in verschiedenen Szenarien}                                         
\label{sec:LaufzeitmessungenbeiThreadsinverschiedenenSzenarien}

    Um die Stärken und Schwächen von \Glspl{vt} im Vergleich zu \Glspl{pt} genauer ermitteln bzw. evaluieren zu können sind Laufzeitmessungen in verschiedenen Szenarien nötig. 
    Anhand dieser Messungen kann ein Leitfaden für die Auswahl des Threadtypen erstellt werden. 
    Folgende Szenarien werden Behandelt:
    \begin{itemize}
        \item Laufzeitmessungen unter hoher Auslastung
        \item Laufzeitmessungen bei sehr vielen Threads und blockierenden (wartenden) Aufgaben.
        \item Laufzeitmessungen bei einer Abhängigkeit zwischen Threads und Verwendung einer geteilten Ressource.
    \end{itemize}


\subsection{Laufzeitmessungen bei Threads unter hoher Auslastung}                                         
\label{subsec:LaufzeitmessungenbeiThreadsunterhoherAuslastung}

    Ziel des der ersten Laufzeitmessung war es herauszufinden wie sich \Glspl{vt} unter ständiger hoher Auslastung schlagen. Dieser Vergleich nutzt keine der von den Entwicklern genannte 
    Stärke von \Glspl{vt}.
    \begin{program} [H]
        \caption{Benchmark eines \Glspl{vt} unter hoher Auslastung}
        \label{prog:BenchmarkEinesVTUnterHoherAuslastung}
    \begin{JavaCode}[language=Java, numbers=left]
        @State(Scope.Benchmark)
public class PerformanceVirtual {

    @Param("10000")
    public int nonsenseIterations;

    @Benchmark
    @BenchmarkMode(Mode.SampleTime)
    @OutputTimeUnit(TimeUnit.MILLISECONDS)
    public void testMethodeVirtual() throws InterruptedException {
        Thread.startVirtualThread(() -> {
            for (int i = 0; i < nonsenseIterations; i++)
                Utility.executeNonesense();
        }).join();
    }
}\end{JavaCode}
    \end{program}
    Dieser Benchmark startet einen \gls{vt} und lässt ihm durchgängige rechenintensive Aufgaben erledigen. Um eine Messung durchführen zu können wartet der Eltern-Thread auf die Beendung der Aufgaben.
    Als Ergebnis wird die gesamte Dauern der Ausführung geliefert. 
