\chapter{Exposé}

\section{Einleitung}

% Der \emph{Deep Space 8K}%
% \footnote{\url{https://ars.electronica.art/center/de/exhibitions/deepspace/}}
% des Ars Electronica Centers in Linz bietet mit seiner $16 \times 9$~m
% großen Projektionsfläche inklusive Positionstracking eine einzigartige
% Möglichkeit, Computerspiele zu realisieren. Diese Spiele verwenden keine
% klassischen Kontrollmechanismen wie Tastatur, Maus oder Gamepad sondern die
% Spieler*innen selbst "steuern" die Inhalte mit ihren Bewegungen. Darüber
% hinaus finden diese Spiele in einem halb-öffentlichen bis öffentlichen Raum
% statt, wodurch sich die Bestimmung der Zielgruppe sowie die Anzahl der
% spielenden Personen schwierig gestaltet. Diese Bachelorarbeit (Masterarbeit)
% beleuchtet diese Problematik und stellt konkrete Lösungsvorschläge anhand
% eines Beispiels dar.

Seit einigen jahren Arbeitet Oracle an Möglichkeiten zur Verbesserung der Skalierbarkeit von Java-Anwendungen im Projekt "Loom".
Der Hauptansatz dieses Vorhabens ist die Einführung von virtuellen Threads, die von der JVM verwaltet werden. 
Diese Threads sind leichtgewichtiger als klassische Plattform Threads und können in größerer Anzahl erzeugt werden. 
In Kombination mit "Continuations" \cite{Continuations}, die es ermöglichen sollen, die Ausführung von Threads zu pausieren und fortzusetzen,
können so bestimmte Anwendungen mit hoher Nebenläufigkeit zukünftig effizienter gestaltet werden. Diese Bachelorarbeit 
beleuchtet diese Technologien und stellt die Neuerungen dem bereits bekanntem gegenüber.





\section{Theoretischer Hintergrund und Stand der Forschung}
\label{sec:hintergrund}

% Large Public Display Games (LPD Games) sind Spiele, die auf großen,
% öffentlich einsehbaren Projektionsflächen dargestellt werden. Derlei
% Installationen finden sich etwa in Museen (wie dem Ars Electronica Center)
% oder auch auf öffentlichen Plätzen. Personen können diese Spiele in der Regel
% jederzeit sehen und auch aktiv an ihnen teilnehmen. Durch diese
% Öffentlichkeit ergeben sich nach \cite{Finke2008} drei Arten von
% Personengruppen, die am Spiel beteiligt sind: \emph{Actors} nehmen aktiv am
% Spielgeschehen teil, \emph{Spectators} verfolgen dieses aktiv und
% \emph{Bystanders} befinden sich lediglich in der Umgebung der öffentlichen
% Installation. Das Ziel ist es, dass Bystanders zur Spectators und Spectators
% zu Actors werden, also das Spiel aktiv spielen. Dieser Prozess soll dabei
% möglichst fließend vonstattengehen und eine größtmögliche Anzahl an Personen
% umfassen. Ein derartiger Ansatz wurde in \cite{Hochleitner2013} als
% \emph{Smooth Transition Gameplay} bezeichnet. Anhand einer konkreten
% Anwendung wird dabei demonstriert, wie dieser Übergang erreicht werden kann,
% es wird jedoch nicht systematisch beschrieben, welche Faktoren dafür nötig sind.

% Einen möglichen Ansatzpunkt bieten dabei die verwendeten Spielmechaniken. Der
% in \cite{Schell2019} aufgestellten Kategorisierung folgend bieten sich hierbei 
% vor allem Mechaniken aus den Bereichen Raum (\emph{Space}), Handlungen
% (\emph{Actions}) und Regeln (\emph{Rules}) an. Dort angesiedelte Mechaniken
% können in einem entsprechenden Gamedesign so eingesetzt werden, dass in einem
% LPD Game die oben genannten Anforderungen -- möglichst einfacher Einstieg und
% gute Skalierbarkeit in Bezug auf die Anzahl der Spieler*innen -- erreicht
% werden.


Die parallele Ausführung von Code bringt immer Schwierigkeiten mit sich. 
Eine davon ist die Bindung von Threads im Betriebssystem an die erstellten Threads auch wenn dieser nur wartet. Bei einer hohen Anzahl an Threads kann dies schnell zu einem Problem werden.
Virtual Threads \cite{oracle21VritualThreads} sollen dieses Problem lösen, indem sie von der JVM verwaltet werden und nicht an Threads im Betriebssystem gebunden sind.
Wartet ein Virtual Thread, wird dieser vom OS-Thread gelöst und dieser kann derzeit Code eines anderen Virtual Threads ausführen.
Dies führt dazu dass bei einer sehr hohen Anzahl an Virtual Threads wesentlich weniger Threads im Betriebssystem erstellt werden müssen und somit die Anwendung vor allem im Server-Clients Bereich
viel skalierbarer wird.
Diese Technologie ist auch die Grundlage für "Structured Concurrency" \cite{structuredConcurrency}. Diese führt StructuredTaskscopes ein,
die die Ausführung von asynchronen Aufgaben erleichtern und verbessern sollen.
Durch die Möglichkeit von den Basisklassen abzuleiten und einzelne Methoden zu überschreiben, können so stark individualiesierte Ausführungsdiesnte entstehen \cite{AsyncStructuredConcurrency}.
Damit sollen Entwickler ihren Code in kleinere, unabhängige Aufgaben oder Aufgabengruppen strukturieren können oder eine Verwendung ohne manuelle Erstellung
und Verwaltung der einzelenen Threads ermöglichen.
Die Technologien soll auch die Erstellung von asynchronen Methoden vereinfachen und die Lesbarkeit des Codes verbessern.
Eine weiters geplante Neuerung sind ScopedValues \cite{JEP481}, die es ermöglichen sollen, unveränderbare Werte in einem bestimmten Scope zu speichern und abzurufen.
Dies gilt auch für alle Child-threads und auch Verschachtelungen von Scopes sind möglich.
Ziel dabei ist eine Steigerung der Robustheit und Verständlichkeit des Codes im Vergleich zu dem bereits bekannten ThreadLocal.






\section{Forschungsfrage}

Aus diesen Ansätzen ergibt sich die folgende Forschungsfrage für diese
Bachelorarbeit:
%
% \begin{quote}
% 	Welche Spielmechaniken müssen auf welche Art und Weise in einem
% 	Game\-design für ein Large Public Display Game eingesetzt werden, um
% 	dieses für eine variable Anzahl von Spieler*innen zu gestalten und diesen
% 	einen möglichst leichten Einstieg zu ermöglichen?
% \end{quote}

\begin{quote}
	Wie und wozu können Virtual Threads und anderere Neuerungen in Projekt Loom verwendet werden und wie unterscheiden sie sich von den bisherigen Technologien?
\end{quote}


\section{Methodik}

% Um diese Frage zu beantworten, soll die Bachelorarbeit (Masterarbeit) als eine
% Kombination von Literaturarbeit und praktischer \bzw prototypischer Umsetzung
% realisiert werden.

% Zunächst soll aus bestehender Literatur (erweiternd zu Abschnitt
% \ref{sec:hintergrund}) erörtert werden, wie mit dem Thema des Smooth
% Transition Gameplay aus Sicht des Gamedesigns umgegangen wurde. Gemeinsame
% Faktoren wie Mechaniken sollen daraus extrahiert werden und als Grundlage für
% ein eigenes, theoretisches Framework dienen. Dieses Framework soll
% schlussendlich eine Liste von Kernmechaniken und Richtlinien für deren
% Anwendung enthalten, sodass LPD Games einen leichten Einstieg sowie eine
% variable Anzahl von Spieler*innen ermöglichen.

% Überprüft soll die Anwendbarkeit dieses Frameworks durch ein eigenes, im
% Rahmen des Semesterprojekts entwickeltes, LPD Game werden. Durch einfache,
% qualitative Fragestellungen an die Spieler*innen und Beobachtungen der
% Besucher*innen während mehrerer Testläufe soll herausgefunden werden, ob der
% Gedanke des Smooth Transition Gameplays mit den verwendeten Mechaniken
% erreicht werden konnte.


Um diese Frage zu beantworten, soll die Bachelorarbeit als eine
Kombination von Literaturarbeit und praktischer \bzw prototypischer Umsetzung
realisiert werden.

Zu Beginn sollen die grundlegenden Eigenschaften und Limitierungen des bestehenden 
Thread-Konzepts untersucht werden um die Motivationen für die Neuerungen zu verstehen. 
Danach soll ein grober Überblick über Projekt Loom und die damit verbundenen Technologien gegeben werden.
Daraufhin sollen die Technologien im Detail erläutert und mit den bisherigen Technologien 
verglichen werden wobei der Schwerpunkt auf Virtual Threads liegt.
Dabei soll auch gezeigt werden wie und in welchen Fällen diese Neuerungen in der Praxis 
angewendet werden können und welche neuen Möglichkeiten sich dadurch ergeben.
Es soll auch verdeutlicht werden welche bestehende Probleme der parallelen Ausführung 
durch die neuen Technologien nicht gelöst werden können.
Die Benchmarks sollen Laufzeit und Speicherverbrauch unter verschiedenen Umständen beinhalten.
Aus diesen Ergebnissen sollen Schlüsse in welchen Fällen die neuen Technologien 
verwendet werden sollten und in welchen nicht.






\section{Erwartete Ergebnisse}

% Als konkretes Ergebnis wird ein Framework aus Spielmechaniken erstellt,
% welches als Grundlage für die Erstellung von LPD Games dienen soll. Es wird
% erwartet, dass sich solche konkreten Mechaniken finden und beschreiben lassen.

% Bei den Tests der praktischen Umsetzung des Frameworks wird ebenfalls eine
% positive Evaluierung erwartet, da es bereits erfolgreiche Konzepte \bzw LPD
% Games gibt, auf deren Erfahrungen aufgebaut werden kann.

Als konkretes Ergebnis wird eine Sammlung Kleinerer Programme erstellt die die neuen Technologien in Projekt Loom demonstrieren.
Es sollen verschiedene Anwendungsfälle jeweils mit den neuen Technologien und den bisherigen Technologien umgesetzt werden sodass die Unterschiede klar erkennbar sind.
Zusätzlich werden die neuen Möglichkeiten erforscht und nach möglichen Problemen untersucht.
Benchmarks mit verschiedenen Rahmenbediengungen sollen die Unterschiede in Laufzeit und Speicherverbrauch möglichst representativ zu zeigen.
Die Implementierung eines großen zusammenhängenden Programms ist nicht geplant, da dies wenig zielführend wäre.
