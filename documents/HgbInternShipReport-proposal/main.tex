%%% Dateikodierung: UTF-8
%%% äöüÄÖÜß  <-- keine deutschen Umlaute hier? UTF-faehigen Editor verwenden!

%%% Magic Comments zum Setzen der korrekten Parameter in kompatiblen IDEs
% !TeX encoding = utf8
% !TeX program = pdflatex 
% !TeX spellcheck = de_DE
% !BIB program = biber

\RequirePackage[utf8]{inputenc} % bei Verw. von lualatex oder xelatex entfernen!
\RequirePackage{hgbpdfa}        % Erzeugt ein PDF/A-2b-konformes Dokument

\documentclass[bachelor,german,smartquotes,proposal,alphabetic]{hgbthesis}
% Zulässige Optionen in [..]: 
%    Typ der Arbeit: 'diploma', 'master' (default), 'bachelor', 'internship'
%		 Zusätzlich für ein Thesis-Exposé: 'proposal' (für 'bachelor' und 'master')
%    Hauptsprache: 'german' (default), 'english'
%    Option zur Umwandlung in typografische Anführungszeichen: 'smartquotes'
%    APA Zitierstil: 'apa'
%%%-----------------------------------------------------------------------------


\graphicspath{{images/}}  % Verzeichnis mit Bildern und Grafiken
\logofile{logo}           % Logo-Datei: images/logo.pdf (kein Logo: \logofile{})
\bibliography{references} % Biblatex-Literaturdatei (references.bib)
\usepackage{hyperref}
\usepackage[acronym]{glossaries}

\makeglossaries

\newacronym{jvm}{JVM}{Java Virtual Machine}
\newacronym{vt}{VT}{Virtual Thread}
\newacronym{pt}{PT}{Plattform Thread}
\newacronym{ot}{OT}{Betriebssystem Thread}
\newacronym{sts}{STS}{StructuredTaskScope}                          % makeglossaries main in der cmd ausführen
\makeglossaries                           % makeglossaries muss dadurch nicht mehr in der cmd ausgeführt werden   

%%%-----------------------------------------------------------------------------
\begin{document}
%%%-----------------------------------------------------------------------------

%%%-----------------------------------------------------------------------------
% Angaben für die Titelei (Titelseite, Erklärung etc.)
%%%-----------------------------------------------------------------------------

% \title{Virtuelle Threads und strukturierte Nebenläufigkeitssteuerung in Java}
% \author{Gernot Hofbauer}
% \programname{Software Engineering}

% \programtype{Fachhochschul-Bachelorstudiengang}
% %\programtype{Fachhochschul-Masterstudiengang} % auswählen/editieren

% \placeofstudy{Hagenberg}
% \dateofsubmission{2024}{09}{26} % {YYYY}{MM}{DD}

% \advisor{FH-Prof. DI Johann Heinzelreiter} % optional
\pagenumbering{alph}                    % damit die Titelseite mit 'a' beginnt und links wieder richtig funktionieren
\includepdf[pages=-]{Titelblatt.pdf}

%%%-----------------------------------------------------------------------------
\frontmatter                                       % Titelei (röm. Seitenzahlen)
%%%-----------------------------------------------------------------------------

% \maketitle
\tableofcontents
\printglossary[type=\acronymtype,title=Akronyme]
%\chapter{Kurzfassung}

    Diese Bachelorarbeit befasst sich mit einigen Neuerungen von \emph{Project Loom} für Java. Eingegangen wird dabei auf \emph{virtuelle Threads}, \emph{StructuredTaskScopes},
    die auch auf virtuellen Threads basieren
    und \emph{ScopedValues}.
    Zu jeder dieser Technologien wird zunächst eine technische Übersicht geboten, bei der darauf eingegangen wird, inwiefern sie sich von den bereits bekannten Technologien unterscheiden.
    Darauf folgt eine eine Erklärung wie sie in Java verwendet werden können, wobei einige Beispiele gezeigt werden. 
    Bei \texttt{StructuredTaskScopes} beinhaltet dies auch das Einbinden eigener Logik und Fehlerbehandlung durch Ableitung von der Basisklasse. Bei \texttt{ScopedValues} wird auch auf die Unterschiede zu \texttt{ThreadLocal} eingegangen. 
    Um eine besseren Vergleich ziehen zu können, wird auch auf die Performance eingegangen. Dazu wurden sechs einfache Benchmarks,
    welche das Laufzeitverhalten untersuchen, erstellt und durchgeführt. Dabei kam das
    „Java Microbenchmark Harness (JMH)“ zum Einsatz. Aufgrund Schwierigkeiten
    beim Messen des Speicherverbrauchs wurde zu diesem Aspekt nur ein Test durchgeführt. Jedes Messergebnis ist mit eine Analyse versehen.
    Auf Basis der Recherche und der Ergebnisse der Benchmarks wird letzten Endes im Fazit eine Zusammenfassung der Stärken und potentiellen Anwendungsszenarien gegeben.		
%\chapter{Abstract}


\begin{english} %switch to English language rules
    This bachelor's thesis deals with some innovations of \emph{Project Loom} for Java. It examines \emph{virtual threads}, \emph{StructuredTaskScopes} that are also based on virtual threads,
    and \emph{ScopedValues}. For each of these technologies, a technical overview is first provided, discussing how they differ from the already known technologies. 
    This is followed by an explanation of how they can be used in Java, with several examples presented. For \texttt{StructuredTaskScopes}, this also includes integrating custom logic and error handling by
    subclassing the base class. For \texttt{ScopedValues}, the differences compared to \texttt{ThreadLocal} are also addressed. In order to enable a better comparison, performance is also examined. 
    For this purpose, six simple benchmarks that investigate runtime behavior were created and executed. In doing so, the “Java Microbenchmark Harness (JMH)” was employed. Due to difficulties in measuring 
    memory consumption, only one test was conducted for this aspect. Each measurement result is accompanied by an analysis. Ultimately, based on the research and the benchmark results, the conclusion provides 
    a summary of the strengths and potential application scenarios.
\end{english}

			


%%%-----------------------------------------------------------------------------
\mainmatter                             % Hauptteil (ab hier arab. Seitenzahlen)
%%%-----------------------------------------------------------------------------

% \chapter{Exposé}

% \section{Einleitung}

% 	Um den Abschluss des Fachhochschul-Bachelorstudiengangs Software Engineering an der FH Hagenberg zu erlangen, muss im Rahmen des Studiums ein zwölfwöchiges
% 	Berufspraktikum absolviert werden. Dieses Praktikum wird in der Regel im sechsten Semester absolviert und erfordert die Zusammenarbeit mit einem von der FH 
% 	geprüften Unternehmen oder einer Einrichtung. Anschließend ist dazu ein Bericht abzugeben, der die Tätigkeiten
% 	und Ergebnisse des Praktikums beschreibt. Es wird beabsichtigt den Bericht vollkommen in deutscher Sprache zu verfassen.

% \section{Unternehmen und Projektumfeld}
% \label{sec:unternehmen}

% 	Das Praktikum wird bei der Firma \emph{ITPRO - Consulting \& Software GmbH} absolviert. Dabei handelt es sich um ein Unternehmen, das sich auf die Entwicklung von Softwarelösungen 
% 	spezialisiert hat. Gegründet wurde sie im Jahre 1999 und hat ihren Hauptsitz in Linz. Kleinere Büros befinden sich in Hagenberg und Ottensheim. Derzeit sind insgesamt um die 
% 	45 Mitarbeiter*innen in den verschiedenen Büros beschäftigt. Diese können frei zwischen den Büros wechseln und arbeiten. Der Großteil des Praktikums wird aus diesem Grund 
% 	in Hagenberg stattfinden.
% 	ITPRO bietet Außendienst-Lösungen, Dienstleistungs-ERP,  Lösungen für ÖPNV (Öffentlicher Personennahverkehr)-Unternehmen und individuelle Software-Lösungen an.

% 	Das Praktikum sieht eine Stelle als Entwickler im internen Team \emph{"Team Nachtschicht"} vor. Dieses besteht derzeit aus 11 Personen und ist für die Entwicklung 
% 	von ÖPNV-Lösungen zuständig. Als Vorgehensmodell des Projekmanagements  wird Scrum verwendet. 
% \section{Aufgabenstellung und Technologien}
% \label{sec:aufgabenstellung}
	
% 	Das Praktikum wird nicht, wie oft üblich als eigenes Projekt durchgeführt sondern als als Unterstützung für ein großes Projekt an dem das Team schon länger arbeitet.
% 	Eine Aufgabenstellung die das gesamte Berufspraktikum umfasst, wurde aus diesem Grund nicht definiert. Die Aufgaben werden vom Betreuer erstellt und sind von unterschiedlichem Umfang.
% 	Meist handelt es sich dabei um ein Feature, das in einem der Module des Projekts implementiert werden soll.
% 	Als Technologien kommt .NET 8 mit dem Entity Framework 8 und Blazor zum Einsatz. Bereits bekannte Aufgaben umfassen die Implementierung verschiedener Blazor-Komponenten,
% 	die eine visuelle Darstellung und Bearbeitung verschiedener Entitäten und Relationen einer Datenbank ermöglichen. Bei der Datenbank handelt es sich um eine SQL-Server-Datenbank,
% 	die ein VDV-452 konformes Datenmodell österreichischer Verkehrsunternehmen abbildet. 
% 	Gelegentlich muss diese Schema auch erweitert oder angepasst werden. 

% 	Beispiele für für typische Aufgaben ist die Erstellung von Blazor-Komponenten für die tabellarische Darstellung von Haltestellen, Linien oder Fahrplänen. Daten sollen 
% 	meist direkt in der Tabelle bearbeitet werden können und für das Einfügen neuer Daten muss ein Formular erstellt werden. Die benötigte Logik und der Datenbankzugriff
% 	ist bei Bedarf ebenfalls zu implementieren. All dies ist auf mehrere Tickets verteilt und zu Beginn einer neuen Aufgabe wird eine Besprechung mit dem Betreuer durchgeführt,
% 	der auch weiterhin als Ansprechpartner bei Fragen zur Verfügung steht. 
% 	Auch größere Aufgaben sind möglich, wie die Erstellung eines Umlaufeditor der die Erstellung von Umläufen aus einzelnen Fahrten durch Drag-and-Drop ermöglicht..
% 	Es ist nicht bekannt welche Aufgaben im Laufe des Praktikums noch anfallen werden.

% \section{Inhalt des Berichts}
% \label{sec:inhalt}

% 	Der Bericht soll die Tätigkeiten und Ergebnisse des Praktikums beschreiben. Dazu wird auf das Unternehmen, die Technologien, Herausforderungen und außergewöhnliche 
% 	Ergebnisse eingegangen. Da es wahrscheinlich nicht zielführend ist auf alle Aufgaben im Detail einzugehen, wird der Bericht die größten und interessantesten Aufgaben
% 	zusammenfassen und darauf gesammelt eingehen. Anhand dieser Aufgaben soll die Vorgehensweise des Praktikanten, die Design-Entscheidungen und die Implementierung der dafür 
% 	relevanten Komponenten beschrieben werden. Zuletzt wird auch über die persönlichen Erfahrungen des Praktikanten berichtet.
% 	Der Vorschlag für ein Inhaltsverzeichnis wird in Form eines Meilensteins nachgereicht. 

% \section{Meilensteine und Zeitplan}
% \label{sec:meilensteine}

% 	Die Erstellung des Berichts wird in mehrere Meilensteine unterteilt, die in einem Zeitplan
% 	dargestellt werden. Die Meilensteine sind wie folgt definiert:

% 	\begin{itemize}
% 		\item Exposee
% 		\item Inhaltsverzeichnis
% 		\item Leseprobe
% 		\item Abgabe der Arbeit zur Korrektur
% 		\item Endgültige Abgabe
% 	\end{itemize}

% 	Der Zeitplan sieht vor, dass die Erstellung des Exposees bis 16.05.2025 eingereicht wird. Eine Woche später erfolgt die Einreichung des vorgeschlagenen Inhaltsverzeichnisses.
% 	Die Erstellung und Abgabe der Leseprobe soll bis 30.05.2025 erfolgen.
% 	Um die eine ausführliche Korrektur der Arbeit zu ermöglichen, wird die Abgabe der Arbeit zur Korrektur bis 15.06.2025 festgelegt.
% 	Die endgültige Abgabe der Arbeit soll bis 30.06.2025 erfolgen.

\chapter{Exposee}

\section{Einleitung}

Um den Abschluss des Fachhochschul-Bachelorstudiengangs Software Engineering an der FH Hagenberg zu erlangen, muss im Rahmen des Studiums ein zwölfwöchiges
Berufspraktikum absolviert werden. Dieses Praktikum wird in der Regel im sechsten Semester absolviert und erfordert die Zusammenarbeit mit einem von der FH 
geprüften Unternehmen oder einer Einrichtung. Anschließend ist dazu ein Bericht abzugeben, der die Tätigkeiten
und Ergebnisse des Praktikums beschreibt. Es wird beabsichtigt, den Bericht vollkommen in deutscher Sprache zu verfassen.

\section{Unternehmen und Projektumfeld} \label{sec:unternehmen}

Das Praktikum wird bei \emph{ITPRO - Consulting \& Software GmbH} absolviert. Dabei handelt es sich um ein Unternehmen, das sich auf die Entwicklung von Softwarelösungen 
spezialisiert hat. Gegründet wurde es im Jahre 1999 und hat seinen Hauptsitz in Linz. Kleinere Büros befinden sich in Hagenberg und Ottensheim. Derzeit sind insgesamt um die 
45 Mitarbeiter*innen in den verschiedenen Büros beschäftigt. Diese können bis auf wenige Ausnahmen frei zwischen den Büros wechseln und dort arbeiten. Der Großteil des Praktikums wird aus diesem Grund 
in Hagenberg stattfinden.
ITPRO bietet Außendienst-Lösungen, Dienstleistungs-ERP, Lösungen für \gls{oepnv}-Unternehmen und individuelle Software-Lösungen an.

Das Praktikum sieht eine Stelle als Entwickler im internen Team \emph{"Team Nachtschicht"} vor. Dieses besteht derzeit aus 11 Personen und ist für die Entwicklung 
von ÖPNV-Lösungen zuständig. Als Vorgehensmodell des Projektmanagements wird Scrum verwendet.

\section{Aufgabenstellung und Technologien} \label{sec:aufgabenstellung}

Das Praktikum wird nicht, wie oft üblich, als eigenes Projekt durchgeführt, sondern als Unterstützung für ein großes Projekt, an dem das Team schon länger arbeitet. Dabei handelt 
es sich um ein \gls{itcs}, das die Planung und Durchführung von Fahrten im öffentlichen Verkehr unterstützt.
Eine Aufgabenstellung, die das gesamte Berufspraktikum umfasst, wurde aus diesem Grund nicht definiert. Die Aufgaben werden vom Betreuer erstellt und sind von unterschiedlichem Umfang.
Meist handelt es sich dabei um ein Feature, das in einem der Module des Projekts implementiert werden soll.
Als Technologien kommt .NET 8 mit dem Entity Framework 8 und Blazor zum Einsatz. Bereits bekannte Aufgaben umfassen die Implementierung verschiedener Blazor-Komponenten,
die eine visuelle Darstellung und Bearbeitung verschiedener Entitäten und Relationen einer Datenbank ermöglichen. Bei der Datenbank handelt es sich um eine SQL-Server-Datenbank,
die ein VDV-452 \cite{VDV452} konformes Datenmodell österreichischer Verkehrsunternehmen abbildet. 
Gelegentlich muss dieses Schema auch erweitert oder angepasst werden. 

Ein Beispiel für typische Aufgaben ist die Erstellung von Blazor-Komponenten für die tabellarische Darstellung von Haltestellen, Linien oder Fahrplänen. Daten sollen 
meist direkt in der Tabelle bearbeitet werden können, und für das Einfügen neuer Daten muss ein Formular erstellt werden. Die benötigte Logik und der Datenbankzugriff
sind bei Bedarf ebenfalls zu implementieren. All dies ist auf mehrere Aufgaben verteilt, und zu Beginn einer neuen Aufgabe wird eine Besprechung mit dem Betreuer durchgeführt,
der auch weiterhin als Ansprechpartner bei Fragen zur Verfügung steht. 
Auch größere Aufgaben sind möglich, wie die Erstellung eines Umlaufeditors, der die Erstellung von Umläufen aus einzelnen Fahrten durch Drag-and-Drop ermöglicht.
Es ist nicht bekannt, welche Aufgaben im Laufe des Praktikums noch anfallen werden.

\section{Inhalt des Berichts} \label{sec:inhalt}

Der Bericht soll die Tätigkeiten und Ergebnisse des Praktikums beschreiben. Dazu wird auf das Unternehmen, die Technologien, Herausforderungen und außergewöhnliche 
Ergebnisse eingegangen. Da es nicht zielführend ist, auf alle Aufgaben im Detail einzugehen, wird der Bericht die größten und interessantesten Aufgaben
zusammenfassen und darauf gesammelt eingehen. Anhand dieser Aufgaben soll die Vorgehensweise des Praktikanten, die Design-Entscheidungen und die Implementierung der dafür 
relevanten Komponenten beschrieben werden. Zuletzt wird auch über die persönlichen Erfahrungen des Praktikanten berichtet.
Der Vorschlag für ein Inhaltsverzeichnis wird separat erstellt.

\section{Meilensteine und Zeitplan} \label{sec:meilensteine}

Die Erstellung des Berichts wird in mehrere Meilensteine unterteilt, die in einem Terminplan
dargestellt werden. Die Meilensteine sind wie folgt definiert:

\begin{enumerate}
    \item Exposee bis 16.05.2025,
    \item Inhaltsverzeichnis bis 23.05.2025,
    \item Leseprobe bis 30.05.2025,
    \item Abgabe der Arbeit zur Korrektur bis 15.06.2025,
    \item Endgültige Abgabe bis 30.06.2025.
\end{enumerate}

Der Zeitplan sieht vor, dass die Erstellung des Exposees bis 16.05.2025 eingereicht wird. Eine Woche später erfolgt die Einreichung des vorgeschlagenen Inhaltsverzeichnisses.
Die Erstellung und Abgabe der Leseprobe soll bis 30.05.2025 erfolgen.
Um eine ausführliche Korrektur der Arbeit zu ermöglichen, wird die Abgabe der Arbeit zur Korrektur bis 15.06.2025 festgelegt.
Die endgültige Abgabe der Arbeit soll bis 30.06.2025 erfolgen.



%%%-----------------------------------------------------------------------------
\MakeBibliography                                           % Quellenverzeichnis
%%%-----------------------------------------------------------------------------
%%%-----------------------------------------------------------------------------
\end{document}
%%%-----------------------------------------------------------------------------
